% !TEX TS-program = pdflatex
% !TEX encoding = UTF-8 Unicode

% This is a simple template for a LaTeX document using the "article" class.
% See "book", "report", "letter" for other types of document.

\documentclass[11pt]{article} % use larger type; default would be 10pt

\usepackage[utf8]{inputenc} % set input encoding (not needed with XeLaTeX)

%%% Examples of Article customizations
% These packages are optional, depending whether you want the features they provide.
% See the LaTeX Companion or other references for full information.

%%% PAGE DIMENSIONS
\usepackage{geometry} % to change the page dimensions
\geometry{a4paper} % or letterpaper (US) or a5paper or....
% \geometry{margin=2in} % for example, change the margins to 2 inches all round
% \geometry{landscape} % set up the page for landscape
%   read geometry.pdf for detailed page layout information

\usepackage{graphicx} % support the \includegraphics command and options

% \usepackage[parfill]{parskip} % Activate to begin paragraphs with an empty line rather than an indent

%%% PACKAGES
\usepackage{booktabs} % for much better looking tables
\usepackage{array} % for better arrays (eg matrices) in maths
\usepackage{paralist} % very flexible & customisable lists (eg. enumerate/itemize, etc.)
\usepackage{verbatim} % adds environment for commenting out blocks of text & for better verbatim
\usepackage{subfig} % make it possible to include more than one captioned figure/table in a single float
% These packages are all incorporated in the memoir class to one degree or another...

%%% HEADERS & FOOTERS
\usepackage{fancyhdr} % This should be set AFTER setting up the page geometry
\pagestyle{fancy} % options: empty , plain , fancy
\renewcommand{\headrulewidth}{0pt} % customise the layout...
\lhead{}\chead{}\rhead{}
\lfoot{}\cfoot{\thepage}\rfoot{}

%%% SECTION TITLE APPEARANCE
\usepackage{sectsty}
\allsectionsfont{\sffamily\mdseries\upshape} % (See the fntguide.pdf for font help)
% (This matches ConTeXt defaults)

%%% ToC (table of contents) APPEARANCE
\usepackage[nottoc,notlof,notlot]{tocbibind} % Put the bibliography in the ToC
\usepackage[titles,subfigure]{tocloft} % Alter the style of the Table of Contents
\renewcommand{\cftsecfont}{\rmfamily\mdseries\upshape}
\renewcommand{\cftsecpagefont}{\rmfamily\mdseries\upshape} % No bold!

%%% END Article customizations

%%% The "real" document content comes below...

\title{Kata04: Data Munging - Questions}
\author{Jared F Bennatt}
%\date{} % Activate to display a given date or no date (if empty),
         % otherwise the current date is printed 

\begin{document}
\maketitle

\begin{itemize}
\item Was the way you wrote the second program influenced by writing the first?

Very much so. Since both were basically the same task just on different data and
formatted slightly differently, the process of parsing each line was the same. 
Likewise finding the ``optimal'' record (day for weather and team for football),
was essentially the same with a minor difference that one was a max and one was 
a min.

\item To what extent did the design decisions you made when writing the original
programs make it easier or harder to factor out common code?

Going back to the first question, it made it fairly straightforward to combine 
the functionality into a single archetype. I wouldn't say it was easy to do, 
but I knew that since I wrote both programs almost identically, it should be 
possible to be done.

\item Is factoring out as much common code as possible always a good thing? Did 
the readability of the programs suffer because of this requirement? How about 
the maintainability?

I think the answer depends on what purpose the codebase is going to serve. The 
\textit{readability} most definitely suffered. I think if you told someone the 
task and then showed them either of the first two programs, anyone with some 
basic knowledge of Java I/O would be able to understand what is going on. The 
problem with the third program, in terms of \textit{readability}, is that there 
are more moving parts and it's less linear. So for instance, you have to 
understand \textit{how do you read a table}? OK, the \texttt{DataParser} can 
read input and give you a table. Except the \texttt{DataParser} is 
\texttt{abstract} so you have to understand how to extend it to use it (I 
actually think looking at the two implementations, \texttt{WeatherDataParser} 
and \texttt{FootballDataParser} are most helpful for understanding how to use 
\texttt{DataParser}).

On the other hand, the \textit{maintainability} is greatly improved in the third
program. I spent quite a bit of time on the weather program because I had to 
figure out what to do, how I was going to do it, write the code, go back and 
decide I should do it differently, run the code, debug the code, etc. The 
football program didn't take nearly as much time, although it still took some 
time because the data was slightly differently formatted, so I had to adapt what
I had done for the weather data, and then duplicate a lot of the code I wrote 
for the weather data. The DRY Fusion on the other hand, while took considerably 
longer to write than either of the first two, was \textit{considerably} faster 
to adapt. So let's say I spent 3 hours writing the third program and getting the
weather data to work. After the weather data worked, I then wrote the part for 
the football data, which probably took 10 minutes to do and get working because 
basically all I needed to do was tell the \texttt{FootballDataParser} what the 
labels were (and type), what delimiter to use and then I had to tell the 
\texttt{FootballAnalyzer} what measure to use and how to compare it. 

So is factoring out common code a good thing? First, I think the answer is 
always yes, to some extent. You \textit{certainly} do not want duplicated code 
(as in duplicated verbatim) because that's going to cause problems if that 
duplicated code ever needs to be refactored (because you'll have to remember the
other place it's duplicated and make changes there as well). Is what I did with 
DRY Fusion always the best idea? I think no. I probably went a little overboard 
with abstraction and I think that makes this overkill for such a simple project 
and I think readability suffers more than it needs to. That is--\textit{unless} 
we are expecting many new data sets and we want something capable of adapting to
many different data sets that are formatted differently. Then I think what I did
is very practical; yes there is an upfront cost in terms of learning the 
codebase, however once you are familiar, you can quickly adapt to many different
data sets quickly.

In conclusion, if you know this is a ``one time'' (or maybe even two or three) 
time program that's going to be run for a very specific task and adapting to new
data isn't foreseen to be an issue, I would say it's better to maintain 
readability. If however you know that the function of the program will be needed
often and for many different types of input, then I think trying to abstract 
away (or factor out) as much functionality as possible is appropriate (and 
should be strived for).
\end{itemize}

\end{document}
